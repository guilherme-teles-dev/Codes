\documentclass[a4paper,12pt]{article}
\usepackage[utf8]{inputenc}
\usepackage[T1]{fontenc}
\usepackage{enumitem}
\usepackage{geometry}

\geometry{left=2.5cm, right=2.5cm, top=2.5cm, bottom=2.5cm}

\title{Exemplo Prático de Serviço de TI: Implementação de um Novo Centro de Atendimento}
\author{Guilherme Brugeff Teles}
\date{}

\begin{document}

\maketitle

\section*{Descrição do Serviço de TI}
O serviço de TI escolhido é a \textbf{implementação de um novo centro de atendimento (Help Desk)} em uma organização de médio porte. Este centro de atendimento servirá como ponto central para todas as solicitações de suporte técnico dos usuários finais. O objetivo é melhorar a eficiência na resolução de incidentes, gerenciar melhor os problemas recorrentes e fornecer suporte técnico proativo, elevando a satisfação dos usuários e a continuidade dos serviços.

\section*{Processos da Transição de Serviços}

\subsection*{1. Planejamento e Suporte à Transição}
\begin{itemize}
    \item \textbf{Atividades:} Planejamento detalhado da transição do novo centro de atendimento, incluindo cronograma, recursos necessários, e definição dos papéis e responsabilidades.
    \item \textbf{Entregas:} Documentação do plano de transição, cronograma do projeto, matriz de responsabilidades, e plano de gestão de riscos.
\end{itemize}

\subsection*{2. Gerenciamento de Mudanças}
\begin{itemize}
    \item \textbf{Atividades:} Identificação, avaliação, e aprovação das mudanças necessárias para a implementação do centro de atendimento.
    \item \textbf{Entregas:} Registros de requisição de mudança (RFC), aprovação das mudanças, e relatórios de impacto e risco.
\end{itemize}

\subsection*{3. Gerenciamento de Ativos e Configuração de Serviço}
\begin{itemize}
    \item \textbf{Atividades:} Identificação, controle e monitoramento dos ativos de TI necessários para o centro de atendimento.
    \item \textbf{Entregas:} CMDB atualizada, registros de ativos de TI, e mapas de configuração.
\end{itemize}

\subsection*{4. Gerenciamento de Liberação e Implantação}
\begin{itemize}
    \item \textbf{Atividades:} Liberação e implantação dos componentes necessários para o novo centro de atendimento.
    \item \textbf{Entregas:} Pacotes de liberação, scripts de implantação, relatórios de validação e verificações pós-implantação.
\end{itemize}

\subsection*{5. Validação e Testes de Serviço}
\begin{itemize}
    \item \textbf{Atividades:} Realização de testes para assegurar que o novo centro de atendimento atende aos requisitos de serviço definidos.
    \item \textbf{Entregas:} Relatórios de testes, checklists de validação, e documentação de incidentes ou problemas identificados durante os testes.
\end{itemize}

\subsection*{6. Avaliação de Mudanças}
\begin{itemize}
    \item \textbf{Atividades:} Avaliação detalhada para analisar o desempenho do novo serviço após a implantação.
    \item \textbf{Entregas:} Relatório de avaliação de mudança, lições aprendidas, e recomendações para melhorias futuras.
\end{itemize}

\subsection*{7. Gerenciamento do Conhecimento}
\begin{itemize}
    \item \textbf{Atividades:} Captura e documentação do conhecimento necessário para operar e manter o centro de atendimento.
    \item \textbf{Entregas:} Documentação do conhecimento, manuais de procedimentos, base de conhecimento acessível para os operadores do Help Desk.
\end{itemize}

\end{document}
